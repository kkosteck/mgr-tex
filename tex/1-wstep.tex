\newpage % Rozdziały zaczynamy od nowej strony.
\section{Wstęp}
W ostatnich latach, wraz z rozwojem technologii cyfrowej i internetu, liczba dostępnych obrazów, zdjęć i diagramów znacząco wzrosła. Zarówno w mediach społecznościowych, stronach internetowych, jak i w różnych obszarach działalności, takich jak medycyna, nauka czy przemysł, obrazy stanowią istotny element komunikacji i przekazu informacji. Jednak człowiek napotyka trudności w efektywnym przetwarzaniu i wykorzystywaniu tej ogromnej ilości wizualnych danych. Analiza i interpretacja obrazów są zadaniem wymagającym znacznego nakładu czasu i wysiłku intelektualnego. Człowiek musi poświęcić wiele uwagi i sił, aby opisać, co~przedstawia dany obraz, uwzględniając detale, kontekst, emocje czy inne istotne elementy. Ten proces może być jeszcze bardziej uciążliwy w przypadku dużych zbiorów wizualnych danych, gdzie manualne generowanie opisów staje się nieefektywne i czasochłonne. W odpowiedzi na te wyzwania, rozwój technik sztucznej inteligencji i uczenia maszynowego otwiera nowe perspektywy w automatycznym generowaniu podpisów do obrazków. Algorytmy uczenia maszynowego pozwalają na tworzenie modeli komputerowych, które są w stanie analizować obrazy, wyodrębniać z nich cechy, rozpoznawać obiekty i kontekst, a następnie generować opisy tekstowe na podstawie tej wiedzy. Dzięki temu automatyczne generowanie podpisów staje się realnym rozwiązaniem, które może przyspieszyć i ułatwić proces analizy obrazów. Oprócz oszczędności czasu i wysiłku, automatyczne generowanie podpisów do obrazków ma również potencjalne znaczenie dla osób niewidomych. Dla nich dostęp do informacji wizualnych jest ograniczony lub niemożliwy. Generowanie tekstowych opisów obrazów staje się zatem narzędziem, które umożliwia im lepsze zrozumienie i korzystanie z wizualnych treści. Dzięki automatycznym podpisom, osoby niewidome mogą otrzymać opisy obrazów, które pośrednio przekazują informacje wizualne, pozwalając im na lepszą orientację i uczestnictwo w kulturze wizualnej.
\subsection{Cel i motywacja pracy}
Celem niniejszej pracy było sprawdzenie skuteczności oraz efektywności aktualnych rozwiązań wykorzystywanych do automatycznego generowania opisów obrazów cyfrowych. Szczególna uwaga została zwrócona na poziom wykorzystywanych zasobów komputerowych w stosunku do poprawy samej skuteczności generowania opisów wraz z uwzględnieniem czasu treningowego. Narastającym problemem coraz bardziej rozwijających się algorytmów uczenia maszynowego jest ich rosnące zapotrzebowanie na zasoby komputerowe. Wraz z postępem w dziedzinie sztucznej inteligencji, modele oparte na głębokim uczeniu stają się coraz bardziej skomplikowane i wymagają większej mocy obliczeniowej oraz większych ilości danych treningowych. Jednym z głównych czynników wymagań zasobów komputerowych jest rozmiar i złożoność sieci neuronowych. Modele, takie jak splotowe sieci neuronowe czy rekurencyjne sieci neuronowe, mogą składać się z setek tysięcy lub nawet milionów parametrów. Trenowanie tych modeli wymaga dużych zbiorów danych oraz intensywnego obliczeniowo procesu optymalizacji tych parametrów. Kolejnym aspektem jest wykorzystanie sprzętu. W przypadku dużych i złożonych sieci neuronowych konieczne jest korzystanie z zaawansowanych jednostek przetwarzania graficznego lub specjalizowanych procesorów tensorowych, które są bardziej efektywne w wykonywaniu operacji macierzowych charakterystycznych dla uczenia maszynowego. Korzystanie z takiego specjalistycznego sprzętu może stanowić wyzwanie ze względu na wysokie koszty i dostępność. Ponadto w samych publikacjach dotyczących proponowanych rozwiązań często nie ma zawartych informacji o efektywności architektury w kontekście zasobów komputerowych.
